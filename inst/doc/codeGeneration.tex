\documentclass{article}
\usepackage{fullpage}
\usepackage{times}

\begin{document}

\section{C++}

Consider the  example  code in myClass.cpp.

We have a class `A', and another `B' that extends `A' and a second class
`Other' via multiple inheritance.


There are two types of interfaces we potentially need
for an arbitrary native data structure, such as a struct.
One is an ability to simply refer to it.
And a second is to be able to copy it to and from R,
field by field.
For C++ classes, the state may not be directly representable
via its fields alone. There may be some underlying, dynamic
computations. So, in this case it makes more sense to focus
on the reference approach.

If possible, we want to avoid duplicating the class
hierarchy and method dispatching in R so that it is
done twice at run time with generic functions in R and
virtual methods in C++.
Rather

To generate an R interface to these classes,
we need to define a new class for this object.
It should extend, either directly or indirectly, the base class
\SClass{RC++Reference}. This is a simple class that
contains an \SClass{externalptr} object by which we refer to the 
native object.

If the C++ class extends another C++ class, then we would like
to have that information.


For each class, we loop over each public method.
We generate a C routine that mirrors
the class method and takes an additional argument
which is the class instance/object, i.e. the this.
The C routine marshals the values from the
R values to their equivalent versions in C/C++.
It then calls the  class method.
Then the result is converted to an R object
and control returned to R.

In addition to the C routine, we create an
R function. This is a simple wrapper to the C
routine and initially, it is responsible for
checking the types of the arguments, 
coercing them to the appropriate types
and invoking the C routine.


When the C++ routine is invoked, the
C++ method dispatch system will 
find the appropriate method within the
class hierarchy.   Ideally, we do not want
to mimic this on the R side.
Suppose we have two classes, A and B where
B extends A. Furthermore, suppose 
A defines  virtual methods named foo
and bar, and a non virtual method 
fixed.
We would therefore define
an R  function and associated C routine
for A::foo, A::bar and A::fixed.
Let's say the R functions are named
foo.A, foo.A and fixed.A
(The similarity to S3 methods is coincidental
at this part in the argument.)
Do we need additional definitions for B?
No. 
A call to  foo.A
with a reference to an instance of class B in C++
\begin{verbatim}
foo.A( myB)
\end{verbatim}
will end up with a call
to the method  ref->foo()
And the C++ virtual table/method lookup mechanism
will find B's own version of that method or use
that inherited from A.

Before making the call to the C routine R_foo_A,
the R code needs to verify that the myB object is
indeed compatible with the underlying foo method
for A. So it needs to verify that the reference is
to an instance of A or a derived class, i.e. B.

There are two approaches to this.
One is to do this in R.
The other is to leave this to C++.
We will want both approaches to be available
and depend on the underlying C++ library when 
possible. For example, the wxWidgets library
provides run-time checking and it is likely
that it is best to leave it to that library.
For C++ code that does not provide such checking,
we can provide it ourselves in R.

A simple approach is to use the S3 model
of dynamic classes.
Each object maintains a vector of class
names. These are the class name of the instance
and all the base class names, and all theirs, etc.
These are computed statically when the interfaces are 
generated and added to the R/C code that
creates instances of the C++ classes.
This is meta data and can be stored like the S4
classes in R.
Then, when validating the class of a reference instance 
to see if it is compatible with the target class,
we need only perform a calculation such as
 "A" \%in\% classNames(b)


\note{Merge across packages, libraries, etc. that extend the native classes.}

See the RDCOM bindings mechanism also.



The C++ and Java syntax for calling an object's method
is familiar and convenient for many:  obj.foo(...)
or obj->foo().
In R, there is no corresponding
"method in object" syntax.
We introduced on in numerous packages over
the years of the form
  obj$foo(...)

obj is an R object which extends a base class, say RC++Reference.
Then, we define a method for the $ operator of that general class.  So
obj$foo returns a function which can then be invoked as obj$foo(a, b,
c) making the entire mechanism seem similar to the the Java/C++
syntax.

There are two ways for this to work with the generated bindings.
One is to run an entirely generic function
which has access to the original object and method name,
i.e. obj and "foo" in our  example.
The function takes an arbitrary number of arguments
and when invoked, finds the relevant method
based on obj, "foo", and the collection of arguments.
This is perfectly acceptable, but delegates a lot of the work
to run-time. The resolving of the method must be
done when the intermediate function is invoked.


When the method is not polymorphic, i.e. there is only one
of them in the class of interest.
In that case, it would be better to return that to the 
user with information about the argument names.
A user may interactively chose  to look at the resulting function
and its arguments.

At its simplest, if there was a collection of functions
indexed by method name. For non-polymorphic methods,
the entry in the table/list would be a function
with well defined parameters that matched the corresponding
C routine and C++ method.
For polymorphic methods, there would be a table of R functions.
The $ operator would return a function that had access
to this table and which would restrict its search for
the appropriate R function to this
subset.

One drawback to this approach is that we have to manage
the collection of functions, and we have to merge
them across base classes.
We may end up consuming more memory, but
we will need experiments to determine this.

For example, in our myClass.h example,
Top would have a table something like

list(i = list(function(val) {}, 
              function() {}),
     ambiguity = list(function() {},
                      function(x = 1) {},
                      function(x = 21.9567) {}))

Next would merge the table from Top 
into its own table

list(i = function(val) {},
     upToDate = function() {})


The resulting table would be

list(i = list(function(val) {},  # from Next
              function() {}),
     ambiguity = list(function() {},
                      function(x = 1) {},
                      function(x = 21.9567) {}),
     upToDate = function() {}    # From Next
    )

Then, obj$name extracts the element from
this list.
If it is a function, the operator function returns that.
If it is a list, the call returns a new function
that has access to this list of specific functions
for that method name. The signature of the new function
can be created by merging the parameters of the 
functions in this list if that is meaningful,
or alternatively default to ...


How does Next's i method call Top's corresponding i method?
Do we have to generate many C routines that implement this
and then have a mechanism in R to access these.


\section{Polymorphic Methods}

C++, unlike C, allows methods or routines to
be overloaded using different signatures.
The return type is the same for all the different
methods, and only the number and type of arguments
change.

One approach is to collect the different wrapper
function definitions into a single list.
Then, we make this available (as a local variable
or via an environment) to a "generic" function that 
takes care of dispatching to the correct function.



One approach to creating this generic function is to use R's class
mechanims.  The S3 dispatch is single-dispatch and so cannot be used
to match multiple arguments.  We can use the S4 mechanism.  We have to
explicitly define classes for each parameter type.  We also may want
or need to determine which method to call dynamically. For example, S4
cannot differentiate between int and int * and will have a single
method for integer.  We can use S4 to get this far and then
differentiate between the two competing functions at run time.

However, this example also indicates that a user
may need to be able to differentiate between the different
overloaded functions.
For example, given a scalar value, i.e. vector of length 1,
she might want to call  foo(double *) rather than
foo(double).
We can make the individual, overloaded functions available
directly, or indirectly via a call to the generic function.
For instance, treating the generic
function as a list that has these functions in its environment
or as an attribute, we could have 
   foo$`doublep`

Similarly, if we have two C++ methods
such as 
 foo(double)
and
 foo(unsigned int)
we cannot automatically differentiate between the two
as both target types correspond to a  numeric
scalar in R.

What do references to a class map to in R types for the signature?



An alternative approach is to build our own dispatch system.  This
downside is that it may be slow and won't take advantage of
improvements in the S4 mechanism.




\section{Enumerations}
These ideas are taken from the SWinTypeLibs and RDCOM approach.  The
idea is that we define an R named integer vector, i.e. with named
elements, that gives the mapping from the symbolic names to the actual
values.  This is assigned to an R variable named
\textit{type-name}Enum, e.g. ColorEnum.

This is the R definition of the enumeration collection.  But we have
individual values \footnote{Vectors of values later} which are derived
from a basic class named EnumValue.  This is simply a named integer
scalar.  For each new enumeration type, we define a new R class with
that enumeration type's name, e.g. Color.  That class name identifies
the corresponding enumeration definition in R. (This can be easily
relaxed.)

We also define methods for converting a character string and an
integer to this type of object.  These perform the relevant checks to
ensure that the value is valid relative to the enumeration definition.


What about anonymous/unnamed enumerations?

\section{Extending C++ class with R methods}
Optionally, we can define a new C++ class that extends the original
C++ class, say myClass, with a new on named, say R_myClass.  This
extends myClass and also a fixed class named R_dynamicMethodClass.
This latter class is a utility class that we use to manage a table of
R functions which can be used to implement C++ methods for the newly
derived class, i.e. R_myClass.



\section{R_dynamicMethodClass}

We have two of these classes.  The first one is used for cases where
there are no polymorphic methods. Each element in the function table
corresponds directly to a separately named method in the C++ class.
When we look to see if there is an R function to implement a C++
method, we need only look up that single name.

When there are polymorphic methods, then we use a differently
structured table.  Each element in that case is a list.  And the list
is a collection of functions which are indexed by a name identifying
the particular version of the polymorphic function.

Precisely how we map the signature list to a string is not
terrifically important from a technical perspective.  It is important
that the R programmers can easily use the mapping when setting
methods.  Using the Java notation for different types and classes is
one approach, or explicit type names, or the number of arguments, etc.


\begin{verbatim}
class R_dynamicMethodClass {

 protected:
 
   /* A hash table for looking up R functions. */

 
}
\end{verbatim}

\end{document}
